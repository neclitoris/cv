\documentclass{template}

\rutrue

\setname{\en{Nikita}\ru{Никита}}{\en{Solodovnikov}\ru{Солодовников}}
\setaddress{\en{Moscow}\ru{Москва}}
\setmobile{}
\setmail{nas140301@gmail.com}
\setlinkedinaccount{} %you can play with color of the template (red is also nice..)
\setgithubaccount{https://github.com/neclitoris} %you can play with color of the template (red is also nice..)
\setthemecolor{red} %you can play with color of the template (red is also nice..)

\begin{document}
%Set variables
%You can add sections, texts, explanations just by copying the style below. Replace the dummy texts "\lipsum[1][x-x]\par" with actual texts.
%Create header
\headerview
\vspace{1ex}
%Sections
%
% Summary
\addblocktext{\en{Summary}\ru{Вкратце}}{%
  \en{I'm a C++ developer with some experience of Python and a big passion for functional programming/Haskell. My favorite topics include distributed systems (with specialized education) and compilers (as an amateur).}
  \ru{Я --- программист на C++ с некоторым опытом на Python и большой любовью к функциональному программированию на Haskell. Мои основные темы --- распределённые системы (проходил эту специализацию в вузе) и компиляторы (любимая мной тема).}
}
%
%Education
\section{\en{Education}\ru{Образование}}
\datedexperience{\en{Higher School of Economics}\ru{ФКН ВШЭ}}{2018--\en{present}\ru{н.в.}}
\explanation{\en{B.S in Applied Mathematics and Informatics}\ru{Бакалавриат: Прикладная математика и информатика}}
     \explanationdetail{
     \en{
         \textbf{\emph{Specialization:}} Distributed systems
     }\ru{
         \textbf{\emph{Специализация:}} Распределённые системы
     }
     }
%
% Experience
 \section{\en{Experience}\ru{Опыт}}
    %
 \datedexperience{\en{Yandex}\ru{Яндекс}}{\en{Summer}\ru{лето} 2019}
    \explanation{\en{C++ intern}\ru{Стажёр, C++}}
    \explanationdetail{
        \coloredbullet \ \en{Distributed systems}\ru{Распределённые системы}
        \coloredbullet \ \en{Build systems}\ru{Системы сборки}
    }
    %
    \datedexperience{Kaspersky}{\en{Fall}\ru{осень} 2020--\en{Spring}\ru{весна} 2022}
    \explanation{\en{Haskell/research intern}\ru{Стажёр, Haskell}}
    \explanationdetail{
        \coloredbullet \ \en{Compilers}\ru{Компиляторы}
    }
    %
    \datedexperience{\en{Yandex}\ru{Яндекс}}{\en{Summer}\ru{лето} 2022--\en{Spring}\ru{весна} 2023}
    \explanation{C++ junior}
    \explanationdetail{
     \coloredbullet \ \en{Distributed systems}\ru{Распределённые системы}
     \coloredbullet \ Geo
    }
%
% Skills
\section{\en{Skills}\ru{Навыки}}
    %
    \newcommand{\skillone}{\createskill{\en{Programming Languages}\ru{Языки программирования}}{\textbf{\emph{\en{Experienced:}\ru{Основной опыт:}}} \ \  C++ \cpshalf \ Haskell \ \ \textbf{\emph{\en{Familiar:}\ru{Знаком:}}} \ \  Javascript \cpshalf \ Python \cpshalf \ Bash }}
    %
    \newcommand{\skilltwo}{\createskill{\en{Software Development}\ru{Технологии}}{
    \begin{tabular}{@{}l@{}}
        \en{OOP}\ru{ООП} \cpshalf\ \en{FP}\ru{Функциональное программирование} \cpshalf\ GIT \cpshalf\ CLI \cpshalf\ Linux \\ \en{Distributed systems}\ru{Распределённые системы}}
    \end{tabular}}
    %
    \newcommand{\skillthree}{\createskill{\en{Frameworks \ \& \ Libraries}\ru{Библиотеки}}{STL \cpshalf\ Boost \cpshalf\ GDAL \cpshalf\ NumPy}}
    %
    \createskills{\skillone, \skilltwo, \skillthree}
%
% Experience
\section{\en{Extra}\ru{Дополнительно}}
    \newcommand{\extraone}{%
        \en{I also love algebra, and have quite a bit of experience there. \href{https://github.com/neclitoris/groebner}{Here}'s my algebra project written in Haskell for my coursework. It's actually quite impressive --- performance outclasses similar C++ projects, and EDSL it provides is nice (for my taste).}
        \ru{А ещё я люблю алгебру, и даже немного её знаю. \href{https://github.com/neclitoris/groebner}{Вот} мой проект по алгебре, написанный на Haskell для курсовой. Довольно классный --- производительнее и удобнее похожих проектов на C++.}
    }
    \newcommand{\listofextras}{\extraone}
    %
    \createbullets{\listofextras}
%
%Footnote
\createfootnote
\end{document}
